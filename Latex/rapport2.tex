%%% packages %%%
%%%%%%%%%%%%%%%%%%%%%%%%%%%%%%%%%%%%%%%%%%%%%%%%%%%%%%%%%%%%%%%%%%%%%%%%%%%%%%%  
\documentclass[frenchb]{report}
%\usepackage{natbib}
\usepackage[dvipsnames]{xcolor}
\usepackage[french]{babel}
\usepackage{url}
\usepackage[utf8x]{inputenc}
\usepackage{graphicx}
\graphicspath{{images/}}
\usepackage{parskip}
\usepackage{fancyhdr}
\usepackage{fancyvrb}
\usepackage{vmargin}
\usepackage{xcolor}
\usepackage{bbm}
\usepackage{amsmath,amssymb}
\usepackage{amsthm}
\usepackage{dsfont}
\usepackage{stmaryrd}
\usepackage{systeme}
\usepackage{enumitem}
\usepackage{xcolor}
\usepackage{pifont}
\usepackage{textcomp}
\usepackage{calrsfs}
\usepackage[T1]{fontenc}
\usepackage[toc,page]{appendix}
\usepackage{lipsum}
\usepackage{verbatim}
\usepackage{listings}
\usepackage{adforn}
\usepackage{float}
% Les packages necessaires pour faire le pseudo code
%%%%%%%%%%%%%%%%%%%%%%%%%
\usepackage{algorithm}
\usepackage{algorithmic}
%%%%%%%%%%%%%%%%%%%%%%%%%
% Je renomme les commandes en français
%%%%%%%%%%%%%%%%%%%%%%%%%%%%%%%%%%%%%%%%%%%%%%%%%%%%%%%%%%%%%%%
\renewcommand{\algorithmicrequire}{\textbf{Entrée(s) :}}
\renewcommand{\algorithmicreturn}{\textbf{retourner}}
\renewcommand{\algorithmicensure}{\textbf{Initialisation ;}}
\renewcommand{\algorithmicwhile}{\textbf{Tant que}}
\renewcommand{\algorithmicdo}{\textbf{Initialisation}}
\renewcommand{\algorithmicendwhile}{\textbf{fin du Tant que ;}}
\renewcommand{\algorithmicend}{\textbf{fin}}
\renewcommand{\algorithmicif}{\textbf{si}}
\renewcommand{\algorithmicendif}{\textbf{fin du si}}
\renewcommand{\algorithmicelse}{\textbf{sinon}}
\renewcommand{\algorithmicelsif}{\textbf{fin du sinon}}
\renewcommand{\algorithmicthen}{\textbf{alors}}
\renewcommand{\algorithmicthen}{\textbf{Étape E}}
\renewcommand{\algorithmicthen}{\textbf{Étape M}}
\renewcommand{\algorithmicfor}{\textbf{pour}}
\renewcommand{\algorithmicforall}{\textbf{pour tout}}
\renewcommand{\algorithmicto}{\textbf{à}}
\renewcommand{\algorithmicendfor}{\textbf{fin du pour}}
\renewcommand{\algorithmicdo}{\textbf{faire}}
\renewcommand{\algorithmicloop}{\textbf{boucler}}
\renewcommand{\algorithmicendloop}{\textbf{fin de la boucle}}
\renewcommand{\algorithmicrepeat}{\textbf{répéter}}
\renewcommand{\algorithmicuntil}{\textbf{jusqu’à}}
%%%%%%%%%%%%%%%%%%%%%%%%%%%%%%%%%%%%%%%%%%%%%%%%%%%%%%%%%%%%%%%
\setlength{\hoffset}{-18pt}        
\setlength{\oddsidemargin}{0pt} % Marge gauche sur pages impaires
\setlength{\evensidemargin}{9pt} % Marge gauche sur pages paires
\setlength{\marginparwidth}{54pt} % Largeur de note dans la marge
\setlength{\textwidth}{481pt} % Largeur de la zone de texte (17cm)
\setlength{\voffset}{-18pt} % Bon pour DOS
\setlength{\marginparsep}{7pt} % Séparation de la marge
\setlength{\topmargin}{0pt} % Pas de marge en haut
\setlength{\headheight}{13pt} % Haut de page
\setlength{\headsep}{10pt} % Entre le haut de page et le texte
\setlength{\footskip}{27pt} % Bas de page + séparation
\setlength{\textheight}{708pt} % Hauteur de la zone de texte (25cm)
\usepackage{hyperref}
\lstset{%
            inputencoding=utf8,
                extendedchars=true,
                literate=%
                {é}{{\'e}}{1}%
                {è}{{\`e}}{1}%
                {à}{{\`a}}{1}%
                {ç}{{\c{c}}}{1}%
                {œ}{{\oe}}{1}%
                {ù}{{\`u}}{1}%
                {É}{{\'E}}{1}%
                {È}{{\`E}}{1}%
                {À}{{\`A}}{1}%
                {Ç}{{\c{C}}}{1}%
                {Œ}{{\OE}}{1}%
                {Ê}{{\^E}}{1}%
                {ê}{{\^e}}{1}%
                {î}{{\^i}}{1}%
                {ô}{{\^o}}{1}%
                {û}{{\^u}}{1}%
                {ë}{{\¨{e}}}1
                {û}{{\^{u}}}1
                {â}{{\^{a}}}1
                {Â}{{\^{A}}}1
                {Î}{{\^{I}}}1
        }
    
    
\lstset{ language=R,
  backgroundcolor=\color{MidnightBlue},   % choose the background color; you must add \usepackage{color} or \usepackage{xcolor}; should come as last argument
   basicstyle=\small\ttfamily\color{white},        % the size of the fonts that are used for the code
  breakatwhitespace=false,         % sets if automatic breaks should only happen at whitespace
  breaklines=true,                 % sets automatic line breaking
  captionpos=b,                    % sets the caption-position to bottom
  commentstyle=\color{SpringGreen},    % comment style
  deletekeywords={data,frame,length,as,character},           % if you want to delete keywords from the given language
  extendedchars=true,              % lets you use non-ASCII characters; for 8-bits encodings only, does not work with UTF-8
  frame=single,	                   % adds a frame around the code
  keepspaces=true,                 % keeps spaces in text, useful for keeping indentation of code (possibly needs columns=flexible)
   keywordstyle=\color{Peach},       % keyword style
  morekeywords={kable,*,...},            % if you want to add more keywords to the set
  numbers=left,                    % where to put the line-numbers; possible values are (none, left, right)
  numbersep=5pt,                   % how far the line-numbers are from the code
  %numberstyle=\tiny\color{gray}, % the style that is used for the line-numbers
  rulecolor=\color{white},         % if not set, the frame-color may be changed on line-breaks within not-black text (e.g. comments (green here))
  showspaces=false,                % show spaces everywhere adding particular underscores; it overrides 'showstringspaces'
  showstringspaces=false,          % underline spaces within strings only
  showtabs=false,                  % show tabs within strings adding particular underscores
  stepnumber=2,                    % the step between two line-numbers. If it's 1, each line will be numbered
      % string literal style
  tabsize=2,	                   % sets default tabsize to 2 spaces
  title=\lstname,                  % show the filename of files included with \lstinputlisting; also try caption instead of title
  mathescape=true,
  escapechar=|
  }
%%%%%%%%%%%%%%%%%%%%%%%%%%%%%%%%%%%%%%%%%%%%%%%%%%%%%%%%%%%%%%%%%%%%%%%%%%%%%%%        


        
\makeatletter
\let\thetitle\@title
\let\theauthor\@author
\let\thedate\@date
\makeatother

%%% commandes mise en page %%%
%%%%%%%%%%%%%%%%%%%%%%%%%%%%%%%%%%%%%%%%%%%%%%%%%%%%%%%%%%%%%%%%%%%%%%%%%%%%%%%        
\newcommand{\ld}{\log_{2}}
\newcommand{\R}{\mathbbm{R}}
\newcommand{\N}{\mathbbm{N}}
\newcommand{\1}{\mathbbm{1}}
\newcommand{\E}{\mathbbm{E}}
\newcommand{\V}{\mathbbm{V}}
\newcommand{\prob}{\mathbbm{P}}
\newcommand{\Nc}{\mathcal{N}}
\newcommand{\Cc}{\mathcal{C}}
\newcommand{\K}{\mathcal{K}}
\newcommand{\Xti}{\widetilde{X_i}}
\newcommand{\Xtj}{\widetilde{X_j}}
\newcommand{\Xn}{\overline{X_n}}
\newcommand{\gn}{\hat{g_n}}
\newcommand{\n}{\mathcal{N}}
\newcommand{\lv}{\mathcal{L}}
\newcommand{\thetat}{\tilde{\theta}}

\newcommand{\console}[1]{\colorbox{black}{\begin{minipage}[c]{1\linewidth}\textcolor{white}{\texttt{#1}}\end{minipage}}}

\newtheorem{prop}{Proposition}
\newtheorem{thm}{Théorème}
\newtheorem{cor}{Corollaire}
\newtheorem{lem}{Lemme}
\newtheorem{hyp}{Hypothèse}
\theoremstyle{definition}\newtheorem{defn}{Définition}
\theoremstyle{definition}\newtheorem{exm}{Exemple}
\theoremstyle{definition}\newtheorem{nota}{Notation}
\theoremstyle{definition}\newtheorem{rem}{Remarque}

\renewcommand{\qedsymbol}{\adfhangingflatleafright}


\begin{document}

%%% Page de garde %%%
%%%%%%%%%%%%%%%%%%%%%%%%%%%%%%%%%%%%%%%%%%%%%%%%%%%%%%%%%%%%%%%%%%%%%%%%%%%%%%%%%%%%%%%%%
\begin{titlepage}
\begin{center}
\includegraphics[scale=0.6]{logo.png}
\hfill
\includegraphics[scale=0.35]{fds_logo.png}\\[3cm]
\linespread{1.2}\huge {\bfseries Projet Master 1 SSD }\\[0.5cm]
\linespread{1.2}\LARGE {\bfseries Un modèle pour les nids de mouettes}\\[1.5cm]
\linespread{1}

{\large Rédigé par\\}
{\Large \textsc{carvaillo} Thomas}\\
{\Large \textsc{côme} Olivier}\\
{\Large \textsc{pralon} Nicolas}\\[1cm]
{\large \emph{Encadrante :} Elodie \textsc{Brunel-Piccinini}}\\[1.5cm] 

\includegraphics[scale=0.7]{imag_logo.png}

\end{center}
\end{titlepage}
%%%%%%%%%%%%%%%%%%%%%%%%%%%%%%%%%%%%%%%%%%%%%%%%%%%%%%%%%%%%%%%%%%%%%%%%%%%%%%%%%%%%%%%%%
\tableofcontents
\newpage
%%%%%%%%%%%%%%%%%%%%%%%%%%%%%%%%%%%%%%%%%%%%%%%%%%%%%%%%%%%%%%%%%%%%%%%%%%%%%%%%%%%%%%%%%

\chapter*{Introduction}

\addcontentsline{toc}{part}{Introduction}


\pagebreak
%%%%%%%%%%%%%%%%%%%%%%%%%%%%%%%%%%%%%%%%%%%%%%%%%%%%%%%%%%%%%%%%%%%%%%%%%%%%%%%%%%%%


%%% Premier chapitre %%%
\chapter{Modélisation du problème}

% section 1
\section{Modélisation du problème}

Dans toute la suite de l'étude on considère seulement des variables aléatoires réelles.\\
Commençons tout d'abord par donner une première définition, qui est au coeur du présent projet.

\begin{defn}[Loi de mélange]
On appelle loi de mélange toute loi dont la densité s'écrit sous la forme d'une combinaison convexe de diverses densités. \\
Soient $X_1, \cdots, X_J$, $J$ var de densité respective par rapport à la mesure de Lebesgue $f_1(x), \cdots, f_J(x)$.
On appel loi de mélange toute fonction $f$ intégrable telle que : 
\begin{align*}
f &:R \rightarrow R_+\\
&:x \mapsto \displaystyle\sum_{i=1}^J \alpha_i f_i(x) ~,~\alpha_i \in\R
\end{align*}

\end{defn}

Plaçons maintenant un cadre, au problème énoncé en introduction, et introduisons les variables aléatoires suivantes :

\begin{itemize}[label=\adfflowerleft]
	\item On définit $X$, la variable aléatoire modélisant la taille des nids
	\item On définit $Z$, la variable aléatoire discrète représentant l'espèce de mouette observé
\end{itemize}

\begin{prop}
La variable $Z$ est discrète et à valeur dans un sous-ensemble fini de $\N$, elle suit donc une loi 
\begin{center} $\displaystyle \sum_{j=1}^J \alpha_j\delta_j$ \end{center}
\underline{où} $J$ représente le nombre d'espèce de mouettes considéré et les $\alpha(j)$ sont des réels, positifs stricts, représentant la proportion de nids de l'espèce $j$, tels que $\displaystyle\sum_{i=1}^J \alpha_j = 1$.\\
\end{prop}

Enfin, nous nous placerons sous les hypothèses suivantes:

\begin{hyp}
Nous supposerons pour chaque espèce que, la taille de leurs nids suit une loi normale.\\
On note $\forall j\in \llbracket 1,J \rrbracket,~\gamma_{\mu_j,v_j}$ sa densité.\\
\end{hyp}

Il s'ensuit de la définition et de l'hypothèse précédente, la proposition suivante.
\begin{prop}
La distribution de la taille des nids de mouettes, \underline{i.e.} $X$, admet pour densité ,au point $x$ et par rapport à la mesure de Lebesgue sur $\R$, la fonction $f_ \theta$ définie comme suit
\begin{center} $f_\theta(x) = \displaystyle\sum_{j=1}^J \alpha_j \gamma_{\mu_j, v_j}(x) $ \end{center}
\end{prop}


\begin{proof}
On vérifie que l'on obtient bien une densité de probabilité, la forme de cette dernière étant la conséquence directe de la définition de la variable aléatoire $X$ : \newline
\begin{center} $\displaystyle\int_\R f_\theta(x) dx = \int_\R\sum_{j=1}^J \alpha_j \gamma_{\mu_j, v_j}(x)dx = \sum_{j=1}^J\alpha_j\int_\R \gamma_{\mu_j, v_j}(x)dx = \sum_{j=1}^J\alpha_j = 1$ 
\end{center}
\end{proof}

\begin{rem}
Puisque que $X$ est intégrale et $Z$ également, on peut généraliser la proposition précédente :\\
$\forall j\in \llbracket 1,J \rrbracket$, la loi de condictionnelle de X sachant que $\big\{Z = j\big\}$ admet pour densité par rapport à la mesure de Lebesgue, la fonction $f_{\theta}(x|Z = j) := \gamma_{\mu_j,v_j}(x)$ $\forall x\in \llbracket \R \rrbracket$.\\
\end{rem}

Il reste maintenant à poser l'hypothèse suivante.
\begin{hyp}
On note $\Theta := \{ \theta = (\alpha_j,\mu_j, v_j)_{1 \leq j \leq J} \text{ tels que } \alpha_j > 0 \text{ } \forall j\in \llbracket 1,J\rrbracket \text{ et } \displaystyle\sum_{j=1}^J\alpha_j=1\}$. \\
On supposera qu'il existe $\theta \in \Theta$ tel que les paramètres de la loi de X sont donnés par $\theta$.\\
\end{hyp}


Le but de ce projet sera d'étudier des méthodes permettant l'estimation des divers paramètres de cette densité. Nous détonerons par $\theta := (\alpha_j, \mu_j, v_j)_{1\leq j\leq J}$ les vecteurs des ces dits paramètres.

\section{Une histoire de densités}

Introduisons une dernière densité et une dernière probabilité, qui nous seront fort utile quant à l'expression des Log-vraisemblances conditionnelles :

\begin{prop} Nous avons les résultats suivant :
\begin{enumerate}
\item La densité du vecteur aléatoire $(X, Z)$ nous est donnée par :
\begin{align*} 
h_\theta : &~\R\times \left\{1,\cdots,J\right\} \rightarrow \R_+\\
&~(x,j) \mapsto \alpha_j\times\gamma_{\mu_j, v_j}(x) 
\end{align*}
\item La loi conditionnelle de $Z$ sachant $\big\{X=x\big\}$ nous est donnée par:
\begin{center} $\prob_\theta(Z = j | X = x)=\displaystyle \frac{\gamma_{\mu(z),v(z)}\times\alpha_z}{\sum_{i=1}^J \alpha_i \times \gamma_{\mu(i), v(i)}(x)}$ \end{center}
\end{enumerate}
\end{prop}

\begin{proof}
Par propriété des lois conditionnelles, nous avons que  
\begin{center} $h_\theta(x,j) = f_\theta(x | Z = j)\times \prob_\theta(Z = j) = f_\theta(x)\times\prob_\theta(Z = j | X = x)$ \end{center}
De ceci, nous déduisons aisément la densité de la loi du vecteur aléatoire $(X,Z)$ : 
\begin{center} $h_\theta(x,j) = \alpha_j\times\gamma_{\mu_j, v_j}(x)$ \end{center}
Puis la densité de la loi conditionnelle de $Z$ sachant $\big\{X = x\big\}$ :
\begin{center}
$\prob_\theta(Z = j | X = x)=\displaystyle \frac{\gamma_{\mu(z),v(z)}\times\alpha_z}{\sum_{i=1}^J \alpha_i \times \gamma_{\mu(i), v(i)}(x)}$
\end{center}
\end{proof}

\begin{rem}
Nous pouvons dès à présent noter que pour un échantillon $X_1, \cdots, X_n$ iid de même loi que $X$, nous avons 
\begin{center}
$\forall i \in \llbracket 1,n \rrbracket$, $h_\theta(X_i,j) = f_\theta(X_i) \times \prob_\theta(Z = j | X = X_i )$
\end{center}
Ceci nous sera utile dans la suite.
\end{rem}

Nous allons dès à présent nous intéresser à l'estimation de ces paramètres.

% Section 2
\section{Première approche du problème}

Afin de résoudre le problème posé, une approche naturelle consiste à déterminer par maximum de vraisemblance les paramètres recherchés. Toute fois cette approche comme nous allons le constater, est difficilement applicable.

On considère dans toute la suite les échantillons $X_1, \cdots, X_n$ iid de même loi que $X$ et $Z_1, \cdots, Z_n$ iid de même que $Z$.\\

On définit $\mathcal{L}_{obs}$ la log-vraisemblance du modèle d'échantillonnage de $X$, nous obtenons ainsi 
\begin{defn} 
La log-vraisemblance des observations s'écrit 
\begin{center} $\mathcal{L}_{obs}(\theta, X_1, \cdots, X_n) := ln\left( \displaystyle\prod_{i=1}^n f_\theta(X_i) \right) = \displaystyle\sum_{i=1}^nln\left( \sum_{j=1}^J \alpha_j \gamma_{\mu_j, v_j}(X_i) \right)$ \end{center}
\end{defn}

Nous voyons dès lors que l'existence d'une expression analytique du maximum de la log-vraisemblance n'est pas assurée. Il est donc nécessaire de trouver un moyen d'approcher les valeurs des différents estimateurs. \newline
Pour ce faire, on peut s'intéresser modèle d'échantillonage du couple $(X,Z)$, car le paramètre $\theta$ du modèle reste inchangé. 

Afin de simplifier les calculs, on se propose de réecrire la fonction de Log-vraisemblance de ce nouveau modèle.

\begin{prop}[Fonction de Log-vraisemblance]
La Log-vraisemblance du modèle s'écrit
\begin{center} $\mathcal{L}_\theta(X_1, \cdots, X_n, Z_1, \cdots, Z_n) = \displaystyle \sum_{j=1}^J |A_j| ln(\alpha_j) + \sum_{j=1}^J\sum_{i\in A_j}ln(\gamma_{\mu_j, v_j}(X_i))$ \end{center}
\underline{où} les $A_j$ sont définis par $A_j := \{ i\in \llbracket1,n \rrbracket$ tels que $Z_i = j \}$ \underline{i.e.} $\displaystyle\bigcup_{j=1}^J A_j = \llbracket1,n \rrbracket$
\end{prop}

\begin{proof}
La Log-vraisemblance du modèle s'écrit:
\begin{align*}
\mathcal{L}_\theta(X_1, \cdots, X_n, Z_1, \cdots, Z_n) &= ln\left(\displaystyle\prod_{i=1}^n h_\theta(X_i, Z_i) \right)\\
&= ln\left(\displaystyle\prod_{i=1}^n  \alpha_{Z_i}\gamma_{\mu_j, v_j}(X_i) \right)\\
&= \displaystyle\sum_{i=1}^n ln(\alpha_{Z_i})+ ln(\gamma_{\mu_j, v_j}(X_i))\\
\end{align*}
$Z_i$ est à valeur dans $\llbracket1,J \rrbracket$, on  partitionne donc $I := \llbracket1,n \rrbracket$ comme $I = \displaystyle\bigcup_{j=1}^J A_j$ pour obtenir
\begin{align*}
\mathcal{L}_\theta(X_1, \cdots, X_n, Z_1, \cdots, Z_n) &=  \displaystyle\sum_{j=1}^J\sum_{i\in A_j} ln(\alpha_{Z_i})+ \sum_{j=1}^J\sum_{i\in A_j} ln(\gamma_{\mu_j, v_j}(X_i))\\
&= \displaystyle\sum_{j=1}^J\sum_{i\in A_j} ln(\alpha_j)+ \sum_{j=1}^J\sum_{i\in A_j} ln(\gamma_{\mu_j, v_j}(X_i)) \\
&= \displaystyle\sum_{j=1}^J |A_j| ln(\alpha_j)+ \sum_{j=1}^J\sum_{i\in A_j} ln(\gamma_{\mu_j, v_j}(X_i)) \\
\end{align*}
\end{proof}
Nous pouvons dès lors maximiser la log-vraisemblance afin d'obtenir les estimateurs souhaités :
\begin{prop}[Estimateurs]
Les estimateurs du maximum de vraisemblance $\hat{\alpha_j}$ (resp. $\hat{\mu_j}$, et $\hat{v_j}$) de $\alpha_j$ (resp. $\mu_j$ et $v_j$) sont donnés par
\begin{center}
$\hat{\alpha_j} = \frac{|A_j|}{n}$
\end{center}
\begin{center}
$ \hat{\mu_j} = \displaystyle\frac{\sum_{i\in A_j} X_i}{|A_j|} $
\end{center}
\begin{center}
$ \hat{v_j} = \displaystyle \frac{\sum_{i\in A_j}(X_i - \hat{\mu_j})^2}{|A_j|}$
\end{center}
\end{prop}

\begin{proof}
Soit $\theta = (\alpha_j, \mu_j, v_j)_{j \in \llbracket 1,J \rrbracket}$. Il s'agit de déterminer 
\begin{center}
	$\underset{\theta\in\R^{3J}, \sum_{j=1}^J\alpha_j = 1}{\text{argmax}}\left(\displaystyle\sum_{j=1}^J |A_j| ln(\alpha_j)+ \sum_{j=1}^J\sum_{i\in A_j} ln(\gamma_{\mu_j, v_j}(x_i))\right)$
\end{center}
Nous avons donc à résoudre un programme de minimisation d'une fonction convexe sur un convexe avec une contraire égalité, il est ainsi naturel de faire appel au Lagrangien. \newline 
Ce dernier s'écrit
\begin{align*} 
L(\theta) &= \displaystyle\sum_{j=1}^J |A_j| ln(\alpha_j)+ \sum_{j=1}^J\sum_{i\in A_j} ln(\gamma_{\mu_j, v_j}(x_i)) - \lambda\times\left(\sum_{j=1}^J\alpha_j - 1\right)\\
&= \displaystyle\sum_{j=1}^J |A_j| ln(\alpha_j)+ \sum_{j=1}^J\sum_{i\in A_j} ln\left(\frac{1}{\sqrt{2\pi v_j}}\exp\left(-\frac{\left(x_i -\mu_j\right)^2}{2v_j} \right)\right) - \lambda\times\left(\sum_{j=1}^J\alpha_j - 1\right)\\
&= \displaystyle\sum_{j=1}^J |A_j|s ln(\alpha_j)+ \sum_{j=1}^J\sum_{i\in A_j}\left( \frac{-1}{2}ln(2\pi v_j) -\frac{(x_i-\mu_j)^2}{2v_j}\right) - \lambda\times\left(\sum_{j=1}^J\alpha_j - 1\right)\\
\end{align*}

Il reste maintenant à résoudre le système suivant, afin d'obtenir le vecteur $\hat{\theta} := (\hat{\alpha_j}, \hat{\mu_j}, \hat{v_j})_{j\in\llbracket 1,J \rrbracket}$ solution du programme.

$
\begin{cases}
\displaystyle\frac{|A_j|}{\hat{\alpha_j}} - \lambda &= 0 \text{ } \forall j \in \llbracket 1,J \rrbracket \\
\displaystyle\sum_{i\in A_j} (x_i-\hat{\mu_j})/\hat{v_j} & = 0 \text{ } \forall j \in \llbracket 1,J \rrbracket \\
\displaystyle\sum_{i\in A_j} \frac{-0.5 \times 2 \times \pi}{2\pi \hat{v_j}} +\frac{(x_i-\hat{\mu_j})^2}{2\hat{v_j}^2} &= 0 \text{ } \forall j \in \llbracket 1,J \rrbracket\\
\displaystyle\sum_{j=1}^J \hat{\alpha_j} = 1
\end{cases}
$

Ceci équivaut à 

$
\begin{cases}
\displaystyle\frac{|A_j|}{\hat{\alpha_j}} &= \lambda \text{ } \forall j \in \llbracket 1,J \rrbracket \\
\displaystyle\sum_{i\in A_j}x_i & =\displaystyle\sum_{i\in A_j} \hat{\mu_j} \text{ } \forall j \in \llbracket 1,J \rrbracket \\
\displaystyle\sum_{i\in A_j} (x_i-\hat{\mu_j})^2 &= \displaystyle\sum_{i\in A_j} \hat{v_j}  \text{ } \forall j \in \llbracket 1,J \rrbracket \\
\displaystyle\sum_{j=1}^J \hat{\alpha_j} = 1
\end{cases}
$
$\Leftrightarrow$
$
\begin{cases}
\displaystyle\frac{|A_j|}{\hat{\alpha_j}} &= \lambda \text{ } \forall j \in \llbracket 1,J \rrbracket \\
\displaystyle\sum_{i\in A_j}\frac{x_i}{|A_j|} & = \hat{\mu_j} \text{ } \forall j \in \llbracket 1,J \rrbracket \\
\displaystyle\sum_{i\in A_j} \frac{(x_i-\hat{\mu_j})^2}{|A_j|} &=  \hat{v_j}  \text{ } \forall j \in \llbracket 1,J \rrbracket \\
\displaystyle\sum_{j=1}^J \hat{\alpha_j} = 1
\end{cases}
$

En sommant les $J$ premières lignes du système, on obtient $\displaystyle\sum_{j=1}^J |A_j| = \sum_{j=1}^J\hat{\alpha_j}\lambda$, \underline{i.e.} $\lambda = n$. En injectant ceci dans le précédent système, on obtient finalement ce qui était annoncé :

$
\begin{cases}
\hat{\alpha_j} &= \displaystyle \frac{|A_j|}{n} \text{ } \forall j \in \llbracket 1,J \rrbracket \\
\hat{\mu_j} &= \displaystyle\sum_{i\in A_j}\frac{x_i}{|A_j|} \text{ } \forall j \in \llbracket 1,J \rrbracket \\
\hat{v_j} &= \displaystyle\sum_{i\in A_j} \frac{(x_i-\hat{\mu_j})^2}{|A_j|}  \text{ } \forall j \in \llbracket 1,J \rrbracket \\
\end{cases}
$
\end{proof}


%section 3
\section{Résolution du problème} % approche réelle // situation naturelle // concordance
L'approche par le modèle d'échantillonnage de $(X,Z)$ étant une bonne manière de résoudre théoriquement le problème, il en est moins quant à sa possible application. Les nids de mouettes pouvant être très similaire, il est très probable que l'observation de l'espèce ayant construit le nid n'est pas aisé à identifier. \\
Il convient donc de déterminer une alternative à notre approche, tout en essayant de rester proche de toute de toute la démarche effectuée dans la section précédente. \\

Une façon de faire cela est d'étudier l'esperance conditionnelle de la Log-vraisemblance du modèle d'échantillonnage sanchant l'observation de l'échantillon $X_1, \cdots, X_n$, elle consitue la meilleur approximation de la log-vraisemblance du modèle, par rapport à ce qui est observable $\underline{i.e}$ $X_1, \cdots, X_n$.\\

\begin{defn}[log-vraisemblance conditionnelle]
On définit la log-vraisemblance $\lv_{c}(\theta, \thetat, X_1, \cdots, X_n) $ conditionnelle par
\begin{center} $\lv_{c}(\theta, \thetat, X_1, \cdots, X_n) = \E_{\thetat}[\lv_\theta(X_1, \cdots, X_n, Z_1, \cdots, Z_n) | X_1, \cdots, X_n]$ \end{center}
\end{defn}

Nous allons maintenant travailler sur l'expression de la log-vraisemblance conditionnelle et en donner une expression simplifiée, qui nous sera fort utile ultérieurement, et une expression plus substancielle, qui nous sera immédiatement utile.

\begin{prop}
Nous avons
\begin{center} $\lv_{c}(\theta, \thetat, X_1, \cdots, X_n) = \displaystyle\sum_{i=1}^n\sum_{j=1}^J ln(h_\theta(X_i ,j))  \prob_{\thetat}(Z = j|X = X_i)$ \end{center}
\end{prop}

\begin{proof}
En effet
\begin{align*}
\lv_{c}(\theta, \thetat, X_1, \cdots, X_n) &= \E_{\thetat}[\lv_\theta(X_1, \cdots, X_n, Z_1, \cdots, X_n)|X_1, \cdots, X_n]\\
&=  \E_{\thetat}\bigg[\displaystyle ln\left(\prod_{i=1}^n  h_\theta(X_i,Z_i)\right) |X_1, \cdots X_n \bigg]\\
&= \displaystyle\sum_{i=1}^n  \E_{\thetat}[ln(h_\theta(X_i,Z_i))|X_1, \cdots, X_n]\\
\end{align*}
Or, les couples $(X_i,Z_i)$ sont indépendants, donc
\begin{align*}
\lv_{c}(\theta, \thetat, X_1, \cdots, X_n) &= \displaystyle\sum_{i=1}^n  \E_{\thetat}[ln(h_\theta(X_i,Z_i))|X_i ]\\
&= \displaystyle\sum_{i=1}^n \sum_{j = 1}^J ln(h_\theta(X_i,j)) \prob_{\thetat} (Z = j|X = X_i)\\
\end{align*}
\end{proof}

Nous nous appuierons sur l'expression suivante pour l'expression des estimateurs du maximum de vraisemblance :

\begin{prop} La fonction $\lv_{c}(\theta, \thetat, X_1, \cdots, X_n)$ se réecrit sous la forme suivante : 
\begin{align*}
 \lv_{c}(\theta, \thetat, X_1, \cdots, X_n) &= -\frac{n}{2}log(2\pi)+\sum_{j=1}^J \left(\sum_{i = 1}^n \prob_{\thetat}(Z= j|X=X_i)\right) log(\alpha_j)\\
&~~~-\frac{1}{2}\sum_{j=1}^J\left(\sum_{i=1}^n  \prob_{\thetat}(Z=j|X=X_i)\left(log(v_j)+\frac{(X_i-\mu_j)^2}{v_j}\right)\right)
\end{align*}
\end{prop}

\begin{proof}
Il suffit de partir de la forme précédente de la log-vraisemblance conditionnelle, on a ainsi : 

\begin{align*}
 \lv_{c}(\theta, \thetat, X_1, \cdots, X_n) &= \sum_{i=1}^n \sum_{j = 1}^J ln(h_\theta(X_i,j)) \prob_{\thetat} (Z = j|X = X_i)\\
&= \sum_{i=1}^n \sum_{j = 1}^J ln(\alpha_j\gamma_{\mu_j,v_j}(X_i))\times \prob_{\thetat} (Z = j|X = X_i)\\
&= \sum_{i=1}^n \sum_{j = 1}^J \left(ln(\alpha_j)+ln(\gamma_{\mu_j,v_j}(X_i))\right)\times \prob_{\thetat} (Z = j|X = X_i)\\
&= \sum_{i=1}^n \sum_{j = 1}^J ln(\alpha_j)\prob_{\thetat} (Z = j|X = X_i)+\sum_{i=1}^n \sum_{j = 1}^J ln(\gamma_{\mu_j,v_j}(X_i))\prob_{\thetat} (Z = j|X = X_i)\\
\end{align*}

Traitons pour commencer la double somme 
\begin{center} $\Delta := \displaystyle \sum_{i=1}^n \sum_{j = 1}^J ln(\gamma_{\mu_j,v_j}(X_i))\prob_{\thetat} (Z = j|X = X_i)$ \end{center}

Nous avons :
\begin{center}
$\gamma_{\mu_j,v_j}(X_i) = \frac{1}{\sqrt{2 \pi v(j)}}e^{-\frac{1}{2}\frac{(X_i - \mu_j)^2}{v_j}}$
\end{center}
\begin{center}
$\prob_{\thetat} (Z = j|X = X_i) = \displaystyle \frac{\alpha_j\gamma_{\mu_j,v_j}}{\displaystyle\sum_{i=1}^J \alpha_j\gamma_{\mu_i,v_i}}$
\end{center}


La double somme devient alors 
\begin{align*}
\Delta &= \sum_{i=1}^n \sum_{j = 1}^J ln\left(\frac{1}{\sqrt{2 \pi v_j}}e^{-\frac{1}{2}\frac{(X_i - \mu_j)^2}{v_j}}\right)\times\prob_{\thetat} (Z = j|X = X_i) \\
&=\sum_{i=1}^n \sum_{j = 1}^J ln\left(\frac{1}{\sqrt{2 \pi v_j}}\right)\times\prob_{\thetat} (Z = j|X = X_i) -\frac{1}{2}\left(\frac{(X_i - \mu_j)^2}{v_j}\right)\times\prob_{\thetat} (Z = j|X = X_i)\\
&=\sum_{i=1}^n \sum_{j = 1}^J -\frac{1}{2} ln(2\pi)\times \prob_{\thetat} (Z = j|X = X_i)-\frac{1}{2}ln(v_j)\times \prob_{\thetat} (Z = j|X = X_i) -\frac{1}{2}\left(\frac{(X_i - \mu_j)^2}{v_j}\right)\times\prob_{\thetat} (Z = j|X = X_i) \\
&=-\frac{n}{2}ln(2\pi)  -\frac{1}{2} \sum_{i=1}^n \sum_{j = 1}^J \left(ln(v_j) + \frac{(X_i - \mu_j)^2}{v_j}\right)\times\prob_{\thetat} (Z = j|X = X_i) \\
&=-\frac{n}{2}ln(2\pi)  -\frac{1}{2} \sum_{j=1}^J \left( \sum_{i = 1}^n \prob_{\thetat} (Z = j|X = X_i)\times \left(ln(v_j) + \frac{(X_i - \mu_j)^2}{v_j}\right)\right) \\
\end{align*}

On obtient de fait le résultat espéré :
\begin{align*}
 \lv_{c}(\theta, \thetat, X_1, \cdots, X_n) &= -\frac{n}{2}log(2\pi)+\sum_{j=1}^J \left(\sum_{i = 1}^n  \prob_{\thetat}(Z=j|X=X_i)\right) \times ln(\alpha_j)\\
&~~~-\frac{1}{2}\sum_{j=1}^J\left(\sum_{i=1}^n \prob_{\thetat}(Z=j|X=X_i)\times\left(ln(v_j)+\frac{(X_i-\mu_j)^2}{v_j}\right)\right)
\end{align*}
\end{proof}

Nous allons dès à présent énoncer une proposition vitale, celle de l'expression des estimateurs du maximum de vraisemblance de la log-vraisemblance conditionnelle. L'expression de ces derniers seront le pivot de l'algorithme EM, que nous présenterons dans le chapitre suivant.

\begin{prop}Sous l'observation de l'echantillon $X_1, \cdots, X_n$ et à $\thetat$ fixé, la fonction $\theta \mapsto \lv_{c}(\theta,\thetat, X_1, \cdots, X_n)$ admet un unique maximum $\theta_M$ donné par : 
\begin{center}
$\hat{\alpha_j} = \displaystyle\frac{1}{n}\sum_{i=1}^n \prob_{\thetat}(Z=j|X=X_i)$
\end{center}
\begin{center}

$\hat{\mu_j}= \displaystyle\frac{\displaystyle\sum_{i=1}^n X_i\prob_{\thetat}(Z=j|X=X_i)}{\displaystyle\sum_{i=1}^n \prob_{\thetat}(Z=j|X=X_i)}$
\end{center}
\begin{center}

$\hat{v_j} = \displaystyle\frac{\displaystyle\sum_{i=1}^n (X_i -\mu_j)^2 \prob_{\thetat}(Z=j|X=X_i)}{\displaystyle\sum_{i=1}^n\prob_{\thetat}(Z=j|X=X_i)}$
\end{center}
\end{prop}

\begin{proof}
Soit $\theta = (\alpha_j, \mu_j, v_j)$. Il s'agit ici de maximiser la fonction $\theta \mapsto \lv_{c}(\theta,\thetat, X_1, \cdots, X_n)$

Puisqu'il s'agit d'un problème d'optimisation, on introduit le Lagrangien du problème sous la contrainte $\sum_{i=1}^n\alpha(i) = 1$.
On rappel que Le Lagrangien est défini ici par : 
\begin{align*}
L &: \Theta \times R \rightarrow R_+\\
&:(\theta,\lambda) \mapsto \lv_c(\theta,\thetat,X_1, \cdots, X_n) - \lambda(\sum_{i=1}^n\alpha_i-1)
\end{align*}

Nous reprenons ici l'écriture de $\lv_c(\theta,\thetat,X_1, \cdots, X_n)$ donné dans la précédente proposition, nous obtenons ainsi l'expression suivante du Lagrangien
\begin{align*}
L(\theta,\lambda) = &-\frac{n}{2}log(2\pi)+\sum_{j=1}^J \left(\sum_{i = 1}^n  \prob_{\thetat}(Z=j|X=X_i)\right) log(\alpha_j)\\
&-\frac{1}{2}\sum_{j=1}^J\left(\sum_{i=1}^n \prob_{\thetat}(Z=j|X=X_i)\left(log(v_j)+\frac{(X_i-\mu_j)^2}{v_j}\right)\right) - \lambda\left(\sum_{i=1}^n\alpha_i-1\right)
\end{align*}

Le Lagrangien admet un maximum sous la contrainte et ce maximum $\theta^*$vérifie le système suivant :

$
\begin{cases}
\displaystyle\frac{\partial \lv}{\partial \alpha_j} (\theta^*) = \frac{\displaystyle\sum_{i=1}^n \prob_{\thetat}(Z=j|X=X_i)}{v_j} - \lambda &=0 \text{ } \forall j \in \llbracket 1,J \rrbracket \\
\displaystyle\frac{\partial \lv}{\partial \mu_j} (\theta^*) = \displaystyle\sum_{i=1}^n \prob_{\thetat}(Z=j|X=X_i)(-2X_i+2\mu_j) &=0 \text{ } \forall j \in \llbracket 1,J \rrbracket \\
\displaystyle\frac{\partial \lv}{\partial v_j} (\theta^*) = -\displaystyle\frac{1}{2v_j}\sum_{i=1}^n \prob_{\thetat}(Z=j|X=X_i) + \frac{1}{2v_j^2}\sum_{i=1}^n \prob_{\thetat}(Z=j|X=X_i)(X_i-\mu_j)^2 &=0 \text{ } \forall j \in \llbracket 1,J \rrbracket \\
\displaystyle\frac{\partial \lv}{\partial \lambda} (\theta^*) = \sum_{i=1}^n\alpha_i-1 &=0 
\end{cases}
$

Puisque $\thetat$ est fixé et l'échantillon $X_1, \cdots, X_n$ observé la loi conditionnelle $\prob_{\thetat}(Z=j|X=X_i)$ est donc une constante. \\

Le système devient alors :

$
\begin{cases}
\alpha_j =\displaystyle \frac{\displaystyle\sum_{i=1}^n g_{\thetat}(j|X=X_i)}{\lambda} &\text{ } \forall j \in \llbracket 1,J \rrbracket \\
\mu_j =\displaystyle \frac{\displaystyle\sum_{i=1}^n X_i\prob_{\thetat}(Z=j|X=X_i)}{\displaystyle\sum_{i=1}^n \prob_{\thetat}(Z=j|X=X_i)} &\text{ } \forall j \in \llbracket 1,J \rrbracket \\
v_j = \displaystyle\frac{\displaystyle\sum_{i=1}^n (X_i -\mu_j)^2 \prob_{\thetat}(Z=j|X=X_i)}{\displaystyle\displaystyle\sum_{i=1}^n\prob_{\thetat}(Z=j|X=X_i)} &\text{ } \forall j \in \llbracket 1,J \rrbracket \\
\displaystyle\sum_{i=1}^J\alpha_i =1 
\end{cases}
$

Sous la contrainte $\displaystyle\sum_{i=1}^n\alpha_i =1$ et la première équation du système précedent on obtient l'égalité suivante : 

\begin{align*}
\sum_{i=1}^J\alpha_i &= \sum_{i=1}^J \left(\frac{\displaystyle\sum_{i=1}^n \prob_{\thetat}(Z=j|X=X_i)}{\lambda}\right)\\
&=\displaystyle\frac{\displaystyle\sum_{i=1}^J \sum_{i=1}^n \prob_{\thetat}(Z=j|X=X_i)}{\lambda}\\
&=\displaystyle\frac{\displaystyle\sum_{i=1}^n \sum_{i=1}^J \prob_{\thetat}(Z=j|X=X_i)}{\lambda}\\
&=\displaystyle\frac{\displaystyle\sum_{i=1}^n 1}{\lambda}= 1\\
\end{align*}

On en déduit ainsi $\lambda = n$, ainsi que le résultat énoncé.
\end{proof}


Tous ces inesthétiques et fastidieux calculs n'ont pas été effectué en vain. Nous les avons réalisé suite à l'introduction d'une notion nouvelle, celle de la log-vraisemblance conditionnelle; qui elle même à été introduite faute de ne pouvoir obtenir une expression analytique de la log-vraisemblance des observations. Nous allons maintenant tâcher de mettre en application les résultats obtenus dans cette section. \newline

\pagebreak
%%%%%%%%%%%%%%%%%%%%%%%%%%%%%%%%%%%%%%%%%%%%%%%%%%%%%%%%%%%%%%%%%%%%%%%%%%%%%%%%%%%%
% chapitre 2
\chapter{L'algorithme EM}

Dans le présent chapitre, nous nous intéresserons A REMPLIR 


\section{Présentation laconique et pseudo-code}

{\color{red}  présentation textuelle rapide de l'algo: etape E, etape M etc...}

Pour l'implémentation de cet algorithme, nous nous sommes appuyés sur le pseudo-code suivant.

\begin{algorithm}
	\caption{\textbf{L’algorithme EM (Dempster et al., 1977).}}
	\begin{algorithmic}[1]
		\REQUIRE{$N \in \mathbb{N}$, $\widehat{\theta_0} \in \Theta$, un jeu de données $x_1 \dots x_n$;}
		\ENSURE
		\STATE {$k:=1$;}
		\WHILE {$K < N + 1$}
		\STATE {$\text{\textbf{ETAPE E :} \textit{Calculer la fonction }} Q(\theta;\widehat{\theta}_{k-1}) = \frac{1}{n} \sum_{i=1}^{n} \mathbb{E}_{\widehat{\theta}_{k-1}} [log f(X_i,Z_i,\theta)|X_i = x_i]$;}
		\STATE {$\text{\textbf{ETAPE M : }} \widehat{\theta}_k = argmax \hspace{1.5mm} Q(\theta;\widehat{\theta}_{k-1})$;}
		\STATE {$k \leftarrow k+1$;}
		\ENDWHILE
		\RETURN {$\widehat{\theta}_N$;}
	\end{algorithmic}
\end{algorithm}

\newpage
% section 2
\section{Un théorème}

Dans cette concise partie, nous donnons une preuve de la croissance de la log-vraisemblance conditionnelle au fur et à mesure des itérations de l'algorithme EM.

\begin{thm}
Soit $(\theta_k)_{k\in \N}$ la suite de paramètres construite à l'aide de l'algorithme EM. La log-vraisemblance $\lv_{obs}$ des observations vérifie 
\begin{center} $\lv_{obs}(\theta_{k+1}, X_1, \cdots, X_n) \geq \lv_{obs}(\theta_k, X_1, \cdots, X_n)$ \end{center}
\end{thm}

\begin{proof}

Nous allons commencer cette preuve en donnant une autre forme de la log-vraisemblance, dépendant de $\lv_ {obs}(\theta, X_1, \cdots, X_n)$ et d'un terme $\kappa_{\theta,\theta_k}$. Nous avons :
\begin{align*}
\lv_c(\theta, \theta_k, X_1, \cdots, X_n) &=  \displaystyle\sum_{i=1}^n\sum_{j=1}^J ln(h_\theta(X_i ,j))  \prob_{\theta_k}(Z = j|X = X_i)\\
&=  \displaystyle\sum_{i=1}^n\sum_{j=1}^J ln\big[f_\theta(X_i)\times \prob_\theta(Z = j|X=X_i)\big]\prob_{\theta_k}(Z = j|X = X_i)\\
&=  \displaystyle\sum_{i=1}^n\sum_{j=1}^J ln(f_\theta(X_i))\prob_{\theta_k}(Z = j|X = X_i) + \sum_{i=1}^n\sum_{j=1}^J ln(\prob_\theta(Z = j|X=X_i))\prob_{\theta_k}(Z = j|X = X_i)\\
&= \displaystyle\sum_{i=1}^n ln(f_\theta(X_i))\times\underbrace{\sum_{j=1}^J \prob_{\theta_k}(Z = j|X = X_i)}_{=1} + \sum_{i=1}^n\sum_{j=1}^J ln(\prob_\theta(Z = j|X = X_i))\prob_{\theta_k}(Z = j|X = X_i)\\
&= \displaystyle\sum_{i=1}^n ln(f_\theta(X_i)) + \sum_{i=1}^n\sum_{j=1}^J ln(\prob_\theta(Z = j|X=X_i))\prob_{\theta_k}(Z = j|X = X_i) \\
&= \lv_{obs}(\theta, X_1, \cdots, X_n) + \kappa_{\theta,\theta_k}
\end{align*}

Dès lors, on obtient
\begin{align*}
\lv_{obs}(\theta_{k+1}, X_1, \cdots, X_n) - \lv_{obs}(\theta_k, X_1, \cdots, X_n) &= \lv_c(\theta_{k+1}, \theta_k, X_1, \cdots, X_n) - \kappa_{\theta_{k+1}, \theta_k} -  \lv_c(\theta_k, \theta_k, X_1, \cdots, X_n) + \kappa_{\theta_{k}, \theta_k}
\end{align*}

Or, la quantité $\lv_c$ est maximisée en $\theta_{k+1}$ lors de l'étape $M$ de l'algorithme, donc
\begin{center} $ \lv_c(\theta_{k+1}, \theta_k, X_1, \cdots, X_n) - \lv_c(\theta_k, \theta_k, X_1, \cdots, X_n) \geq 0$ \end{center}

Il reste donc à prouver que 
\begin{center} $\kappa_{\theta_{k}, \theta_k}-\kappa_{\theta_{k+1}, \theta_k} \geq 0$ \end{center}

En effet, nous avons
\begin{align*}
\kappa_{\theta_{k}, \theta_k}-\kappa_{\theta_{k+1}, \theta_k} &= \sum_{i=1}^n\sum_{j=1}^J ln(\prob_{\theta_k}(Z = j|X=X_i))\prob_{\theta_k}(Z = j|X = X_i) \\
&- \sum_{i=1}^n \sum_{j=1}^J ln(\prob_{\theta_{k+1}}(Z = j|X=X_i))\prob_{\theta_k}(Z = j|X = X_i) \\
&= \sum_{i=1}^n \sum_{j=1}^J ln\left(\frac{\prob_{\theta_k}(Z = j|X=X_i)}{\prob_{\theta_{k+1}}(Z = j|X=X_i)}\right)\prob_{\theta_k}(Z = j|X = X_i)\\
&= - \sum_{i=1}^n \sum_{j=1}^J ln\left(\frac{\prob_{\theta_{k+1}}(Z = j|X=X_i)}{\prob_{\theta_k}(Z = j|X=X_i)}\right)\prob_{\theta_k}(Z = j|X = X_i)\\
&\geq - \sum_{i=1}^n ln\left(\sum_{j=1}^J \frac{\prob_{\theta_{k+1}}(Z = j|X=X_i)}{\prob_{\theta_k}(Z = j|X=X_i)}\prob_{\theta_k}(Z = j|X = X_i) \right)\\
&\text{$\big[$Cette dernière inégalité est due à la convexité du $log$ et au fait que $\displaystyle\sum_{j=1}^J\prob_{\theta_k}(Z = j|X)=1\big]$}\\
&= - \sum_{i=1}^n ln\left(\sum_{j=1}^J\prob_{\theta_{k+1}}(Z = j|X=X_i)\right)\\
&=  - \sum_{i=1}^n ln(1)\\
&= 0
\end{align*}


On obtient ainsi

\begin{center} $\kappa_{\theta_{k}, \theta_k}-\kappa_{\theta_{k+1}, \theta_k} \geq 0$ \end{center}


Et finalement
\begin{center} $\lv_{obs}(\theta_{k+1}, X_1, \cdots, X_n) \geq \lv_{obs}(\theta_k, X_1, \cdots, X_n)$ \end{center}

\end{proof}

\newpage




%%%%%%%%%%%%%%%%%%%%%%%%%%%%%%%%%%%%%%%%%%%%%%%%%%%%%%%%%%%%%%%%%%%%%%%%%%%%%%%%%%%%
\chapter*{Bibliographie}
\addcontentsline{toc}{part}{Bibliographie}
 
Liens utiles

\url{https://www.lpsm.paris/pageperso/rebafka/BookGraphes/algorithme-em.html} \newline
\url{https://members.loria.fr/moberger/Enseignement/AVR/Exposes/algo-em.pdf} \newline
\url{http://faculty.washington.edu/fxia/courses/LING572/EM_collins97.pdf}\newline
\url{https://core.ac.uk/download/pdf/155777956.pdf}\newline
\url{http://www.cmap.polytechnique.fr/~bansaye/CoursTD6.pdf}


\pagebreak
%%%%%%%%%%%%%%%%%%%%%%%%%%%%%%%%%%%%%%%%%%%%%%%%%%%%%%%%%%%%%%%%%%%%%%%%%%%%%%%%%%%%
\begin{appendix}
\chapter{Annexe}

\end{appendix}

\end{document}