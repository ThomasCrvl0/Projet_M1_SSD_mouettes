\documentclass[frenchb]{report}
%\usepackage{natbib}
\usepackage[dvipsnames]{xcolor}
\usepackage[french]{babel}
\usepackage{url}
\usepackage[utf8x]{inputenc}
\usepackage{graphicx}
\graphicspath{{images/}}
\usepackage{parskip}
\usepackage{fancyhdr}
\usepackage{fancyvrb}
\usepackage{vmargin}
\usepackage{xcolor}
\usepackage{bbm}
\usepackage{amsmath,amssymb}
\usepackage{amsthm}
\usepackage{dsfont}
\usepackage{stmaryrd}
\usepackage{systeme}
\usepackage{enumitem}
\usepackage{xcolor}
\usepackage{pifont}
\usepackage{textcomp}
\usepackage{calrsfs}
\usepackage[T1]{fontenc}
\usepackage[toc,page]{appendix}
\usepackage{lipsum}
\usepackage{verbatim}
\usepackage{listings}
\usepackage{adforn}
\usepackage{float}
\setlength{\hoffset}{-18pt}        
\setlength{\oddsidemargin}{0pt} % Marge gauche sur pages impaires
\setlength{\evensidemargin}{9pt} % Marge gauche sur pages paires
\setlength{\marginparwidth}{54pt} % Largeur de note dans la marge
\setlength{\textwidth}{481pt} % Largeur de la zone de texte (17cm)
\setlength{\voffset}{-18pt} % Bon pour DOS
\setlength{\marginparsep}{7pt} % Séparation de la marge
\setlength{\topmargin}{0pt} % Pas de marge en haut
\setlength{\headheight}{13pt} % Haut de page
\setlength{\headsep}{10pt} % Entre le haut de page et le texte
\setlength{\footskip}{27pt} % Bas de page + séparation
\setlength{\textheight}{708pt} % Hauteur de la zone de texte (25cm)
\usepackage{hyperref}
\lstset{%
            inputencoding=utf8,
                extendedchars=true,
                literate=%
                {é}{{\'e}}{1}%
                {è}{{\`e}}{1}%
                {à}{{\`a}}{1}%
                {ç}{{\c{c}}}{1}%
                {œ}{{\oe}}{1}%
                {ù}{{\`u}}{1}%
                {É}{{\'E}}{1}%
                {È}{{\`E}}{1}%
                {À}{{\`A}}{1}%
                {Ç}{{\c{C}}}{1}%
                {Œ}{{\OE}}{1}%
                {Ê}{{\^E}}{1}%
                {ê}{{\^e}}{1}%
                {î}{{\^i}}{1}%
                {ô}{{\^o}}{1}%
                {û}{{\^u}}{1}%
                {ë}{{\¨{e}}}1
                {û}{{\^{u}}}1
                {â}{{\^{a}}}1
                {Â}{{\^{A}}}1
                {Î}{{\^{I}}}1
        }
    
    
\lstset{ language=R,
  backgroundcolor=\color{MidnightBlue},   % choose the background color; you must add \usepackage{color} or \usepackage{xcolor}; should come as last argument
   basicstyle=\small\ttfamily\color{white},        % the size of the fonts that are used for the code
  breakatwhitespace=false,         % sets if automatic breaks should only happen at whitespace
  breaklines=true,                 % sets automatic line breaking
  captionpos=b,                    % sets the caption-position to bottom
  commentstyle=\color{SpringGreen},    % comment style
  deletekeywords={data,frame,length,as,character},           % if you want to delete keywords from the given language
  extendedchars=true,              % lets you use non-ASCII characters; for 8-bits encodings only, does not work with UTF-8
  frame=single,	                   % adds a frame around the code
  keepspaces=true,                 % keeps spaces in text, useful for keeping indentation of code (possibly needs columns=flexible)
   keywordstyle=\color{Peach},       % keyword style
  morekeywords={kable,*,...},            % if you want to add more keywords to the set
  numbers=left,                    % where to put the line-numbers; possible values are (none, left, right)
  numbersep=5pt,                   % how far the line-numbers are from the code
  %numberstyle=\tiny\color{gray}, % the style that is used for the line-numbers
  rulecolor=\color{white},         % if not set, the frame-color may be changed on line-breaks within not-black text (e.g. comments (green here))
  showspaces=false,                % show spaces everywhere adding particular underscores; it overrides 'showstringspaces'
  showstringspaces=false,          % underline spaces within strings only
  showtabs=false,                  % show tabs within strings adding particular underscores
  stepnumber=2,                    % the step between two line-numbers. If it's 1, each line will be numbered
      % string literal style
  tabsize=2,	                   % sets default tabsize to 2 spaces
  title=\lstname,                  % show the filename of files included with \lstinputlisting; also try caption instead of title
  mathescape=true,
  escapechar=|
  }
        
        
\makeatletter
\let\thetitle\@title
\let\theauthor\@author
\let\thedate\@date
\makeatother


\newcommand{\ld}{\log_{2}}
\newcommand{\R}{\mathbbm{R}}
\newcommand{\N}{\mathbbm{N}}
\newcommand{\1}{\mathbbm{1}}
\newcommand{\E}{\mathbbm{E}}
\newcommand{\V}{\mathbbm{V}}
\newcommand{\prob}{\mathbbm{P}}
\newcommand{\Nc}{\mathcal{N}}
\newcommand{\Cc}{\mathcal{C}}
\newcommand{\Xti}{\widetilde{X_i}}
\newcommand{\Xtj}{\widetilde{X_j}}
\newcommand{\Xn}{\overline{X_n}}
\newcommand{\gn}{\hat{g_n}}
\newcommand{\n}{\mathcal{N}}

\newcommand{\console}[1]{\colorbox{black}{\begin{minipage}[c]{1\linewidth}\textcolor{white}{\texttt{#1}}\end{minipage}}}

\newtheorem{prop}{Proposition}
\newtheorem{thm}{Théorème}
\newtheorem{cor}{Corollaire}
\newtheorem{lem}{Lemme}
\newtheorem{hyp}{Hypothèse}
\theoremstyle{definition}\newtheorem{defn}{Définition}
\theoremstyle{definition}\newtheorem{exm}{Exemple}
\theoremstyle{definition}\newtheorem{nota}{Notation}
\theoremstyle{definition}\newtheorem{rem}{Remarque}



\renewcommand{\qedsymbol}{\adfhangingflatleafright}


\begin{document}
%%%%%%%%%%%%%%%%%%%%%%%%%%%%%%%%%%%%%%%%%%%%%%%%%%%%%%%%%%%%%%%%%%%%%%%%%%%%%%%%%%%%%%%%%
\begin{titlepage}

\begin{center}
\includegraphics[scale=0.5]{logo.png}\\[1cm]
{\LARGE Université de Montpellier}\\[1.5cm]
\linespread{1.2}\huge {\bfseries Projet M1 SSD }\\[0.5cm]
\linespread{1.2}\LARGE {\bfseries Un modèle pour les nids de mouettes}\\[1.5cm]
\linespread{1}

{\large Rédigé par\\}
{\Large \textsc{carvaillo} Thomas}\\
{\Large \textsc{côme} Olivier}\\
{\Large \textsc{pralon} Nicolas}\\[1cm]
{\large \emph{Encadrante :} Elodie \textsc{Brunel-Piccinini}}\\[1cm] % if applicable
%\textit{dans le}\\[0.3cm]
%Départment d'Informatique, UFR Sciences et Techniques\\[2cm]
\today
\end{center}

\end{titlepage}
%%%%%%%%%%%%%%%%%%%%%%%%%%%%%%%%%%%%%%%%%%%%%%%%%%%%%%%%%%%%%%%%%%%%%%%%%%%%%%%%%%%%%%%%%
\tableofcontents
\newpage
%%%%%%%%%%%%%%%%%%%%%%%%%%%%%%%%%%%%%%%%%%%%%%%%%%%%%%%%%%%%%%%%%%%%%%%%%%%%%%%%%%%%%%%%%

\chapter*{Introduction}
\addcontentsline{toc}{part}{Introduction}
\lipsum


\pagebreak
%%%%%%%%%%%%%%%%%%%%%%%%%%%%%%%%%%%%%%%%%%%%%%%%%%%%%%%%%%%%%%%%%%%%%%%%%%%%%%%%%%%%

\chapter{Un peu de théorie}

\section{Modélisation du problème}
Afin de modéliser commodément le problème, nous introduisons les variables aléatoires suivantes :

\begin{itemize}[label=\adfflowerleft]
	\item La variable aléatoire $X$, modélisant la taille des nids
	\item La variable aléatoire $Y$, décrivant la taille du nid d'une espèce donnée
	\item Et $Z$, la variable aléatoire représentant l'espèce de mouette qui a construit le nid
\end{itemize}

\begin{hyp}
Nous supposerons que la taille des nids d'une espèce $j$ ( \underline{i.e.} $X$ conditionnelement à $(Z=j)$ ) suit une loi normale $\n(\mu_j,v_j)$. 
\end{hyp}

\begin{prop}
La variable $Z$ est discrète et à valeur dans un sous-ensemble fini de $\N$, elle suit donc une loi 
\begin{center} $\displaystyle \sum_{j=1}^J \alpha(j)\delta_j$ \end{center}
\underline{où} $J$ représente le nombre d'espèce de mouettes considéré et les $\alpha(j)$ sont des réels, positifs stricts, représentant la proportion de nids de l'espèce $j$, tels que $\displaystyle\sum_{i=1}^J \alpha(j) = 1$.
\end{prop}

Il s'ensuit la proposition suivante, qui sera la racine du présent projet.
\begin{prop}
La distribution de la taille des nids de mouettes, \underline{i.e.} $X$, admet pour densité ,au point $x$ et par rapport à la mesure de Lebesgue sur $\R$, la fonction $f_ \theta$ définie comme suit
\begin{center} $f_\theta(x) = \displaystyle\sum_{j=1}^J \alpha(j) \gamma_{\mu_j, v_j}(x) $ \end{center}
\end{prop}

Le but de ce projet sera d'étudier des méthodes permettant l'estimation des divers paramètres de cette densité. Nous détonerons par $\theta := (\alpha_j, \mu_j, v_j)_{1\leq j\leq J}$ les vecteurs des ces dits paramètres. \newline
Pour cela, nous nous placerons sous l'hypothèse suivante

\begin{hyp}
Soit $\Theta := \{ \theta = (\alpha_j,\mu_j, v_j)_{1 \leq j \leq J} \text{ tels que } \alpha_j > 0 \forall j\in \llbracket 1,J\rrbracket \text{ et } \displaystyle\sum_{j=1}^J\alpha_j=1\}$. Soient $X_1, \cdots, X_n$ un échantillon iid de même loi que $X$. On supposera qu'il existe un $\theta \in \Theta$ tel que les données récoltées, ici les tailles des nids, soient la réalisation du précédent échantillon.
\end{hyp}

\begin{nota}[Densités]
Le vecteur $\theta$ ayant été dûment introduit, nous noterons
\begin{enumerate}
	\item $g_\theta(z) = \alpha(z)$ la densité de la variable aléatoire $Z$
	\item $f_\theta(x | Z=j) := \gamma_{\mu_j, v_j}(x)$  la densité de la loi de $X$ sachant $Z$, \underline{i.e.} de la variable aléatoire $Y$
\end{enumerate}
qui sont, respectivement, contre la mesure de comptage sur $\N$, et par rapport à la mesure de $\mathcal{L}$ebesgue sur $\R$,
\end{nota}

Introduisons deux dernières densités, qui nous seront fort utile quant à l'expression des Log-vraisemblances conditionnelles :

\begin{prop}[Densité de $Z$ sachant $X$]
$g_\theta(z | X=x ) = \frac{\displaystyle\alpha(z)\gamma_{\mu_j, v_j}(x)}{\displaystyle\sum_{j=1}^J\alpha(j)\gamma_{\mu_j, v_j}(x)}$, pour $z\in \{1,\cdots,J\}$
\end{prop}

\begin{proof}
En effet, BLABLABLA Nicolas le fera
\end{proof}

\begin{prop}[Densité du vecteur (X,Z)]
Soit le vecteur aléatoire $(X, Z)$; sa densité nous est donnée par
\begin{center} $h_\theta(x,z) := \alpha(z)\gamma_{\mu_j, v_j}(x)$ \end{center}
\end{prop}

\begin{proof}
En effet, si on note par $F$ la fonction de répartition de ce vecteur aléatoire, on obtient
\begin{align*}
F(x, z) &= \prob(X \leq x, Z = z)\\
&= \prob(X\leq x | Z = z)\times\prob(Z=z)\\
&= \int_{-\infty}^x \gamma_{\mu_z,v_z}(t)dt \times \alpha(z)\\
&= \int_{-\infty}^x \int_\N\alpha(z)\gamma_{\mu_z,v_z}(t)dt\delta_z
\end{align*}
La densité de $(X,Z)$ s'ensuit.
\end{proof}

\section{Le cas simple}

\section{Le cas réel}

\pagebreak
%%%%%%%%%%%%%%%%%%%%%%%%%%%%%%%%%%%%%%%%%%%%%%%%%%%%%%%%%%%%%%%%%%%%%%%%%%%%%%%%%%%%

\chapter*{Bibliographie}
\addcontentsline{toc}{part}{Bibliographie}



\pagebreak
%%%%%%%%%%%%%%%%%%%%%%%%%%%%%%%%%%%%%%%%%%%%%%%%%%%%%%%%%%%%%%%%%%%%%%%%%%%%%%%%%%%%
\begin{appendix}
\chapter{Annexe}

\end{appendix}

\end{document}
